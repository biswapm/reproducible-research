% Options for packages loaded elsewhere
\PassOptionsToPackage{unicode}{hyperref}
\PassOptionsToPackage{hyphens}{url}
%
\documentclass[
]{article}
\usepackage{lmodern}
\usepackage{amssymb,amsmath}
\usepackage{ifxetex,ifluatex}
\ifnum 0\ifxetex 1\fi\ifluatex 1\fi=0 % if pdftex
  \usepackage[T1]{fontenc}
  \usepackage[utf8]{inputenc}
  \usepackage{textcomp} % provide euro and other symbols
\else % if luatex or xetex
  \usepackage{unicode-math}
  \defaultfontfeatures{Scale=MatchLowercase}
  \defaultfontfeatures[\rmfamily]{Ligatures=TeX,Scale=1}
\fi
% Use upquote if available, for straight quotes in verbatim environments
\IfFileExists{upquote.sty}{\usepackage{upquote}}{}
\IfFileExists{microtype.sty}{% use microtype if available
  \usepackage[]{microtype}
  \UseMicrotypeSet[protrusion]{basicmath} % disable protrusion for tt fonts
}{}
\makeatletter
\@ifundefined{KOMAClassName}{% if non-KOMA class
  \IfFileExists{parskip.sty}{%
    \usepackage{parskip}
  }{% else
    \setlength{\parindent}{0pt}
    \setlength{\parskip}{6pt plus 2pt minus 1pt}}
}{% if KOMA class
  \KOMAoptions{parskip=half}}
\makeatother
\usepackage{xcolor}
\IfFileExists{xurl.sty}{\usepackage{xurl}}{} % add URL line breaks if available
\IfFileExists{bookmark.sty}{\usepackage{bookmark}}{\usepackage{hyperref}}
\hypersetup{
  pdftitle={PA1\_template},
  pdfauthor={Biswa Pujarini},
  hidelinks,
  pdfcreator={LaTeX via pandoc}}
\urlstyle{same} % disable monospaced font for URLs
\usepackage[margin=1in]{geometry}
\usepackage{graphicx,grffile}
\makeatletter
\def\maxwidth{\ifdim\Gin@nat@width>\linewidth\linewidth\else\Gin@nat@width\fi}
\def\maxheight{\ifdim\Gin@nat@height>\textheight\textheight\else\Gin@nat@height\fi}
\makeatother
% Scale images if necessary, so that they will not overflow the page
% margins by default, and it is still possible to overwrite the defaults
% using explicit options in \includegraphics[width, height, ...]{}
\setkeys{Gin}{width=\maxwidth,height=\maxheight,keepaspectratio}
% Set default figure placement to htbp
\makeatletter
\def\fps@figure{htbp}
\makeatother
\setlength{\emergencystretch}{3em} % prevent overfull lines
\providecommand{\tightlist}{%
  \setlength{\itemsep}{0pt}\setlength{\parskip}{0pt}}
\setcounter{secnumdepth}{-\maxdimen} % remove section numbering

\title{PA1\_template}
\author{Biswa Pujarini}
\date{30/05/2020}

\begin{document}
\maketitle

library(ggplot2) library(plyr) library(``data.table'') \# Load and read
data

fileUrl \textless-
``\url{https://d396qusza40orc.cloudfront.net/repdata\%2Fdata\%2Factivity.zip}''
download.file(fileUrl, destfile = paste0(getwd(),
`/repdata\%2Fdata\%2Factivity.zip'), method = ``curl'')
unzip(``repdata\%2Fdata\%2Factivity.zip'',exdir = ``data'')

activityDT \textless- data.table::fread(input = ``data/activity.csv'')

\#\#Calculate the total number of steps taken per day? 1. Calculate the
total number of steps taken per day

Total\_Steps \textless- activityDT{[}, c(lapply(.SD, sum, na.rm =
FALSE)), .SDcols = c(``steps''), by = .(date){]}

head(Total\_Steps, 10)

\hypertarget{date-steps}{%
\subsection{date steps}\label{date-steps}}

\hypertarget{na}{%
\subsection{1: 2012-10-01 NA}\label{na}}

\hypertarget{section}{%
\subsection{2: 2012-10-02 126}\label{section}}

\hypertarget{section-1}{%
\subsection{3: 2012-10-03 11352}\label{section-1}}

\hypertarget{section-2}{%
\subsection{4: 2012-10-04 12116}\label{section-2}}

\hypertarget{section-3}{%
\subsection{5: 2012-10-05 13294}\label{section-3}}

\hypertarget{section-4}{%
\subsection{6: 2012-10-06 15420}\label{section-4}}

\hypertarget{section-5}{%
\subsection{7: 2012-10-07 11015}\label{section-5}}

\hypertarget{na-1}{%
\subsection{8: 2012-10-08 NA}\label{na-1}}

\hypertarget{section-6}{%
\subsection{9: 2012-10-09 12811}\label{section-6}}

\hypertarget{section-7}{%
\subsection{10: 2012-10-10 9900}\label{section-7}}

\#If you do not understand the difference between a histogram and a
barplot, research the difference between them. Make a histogram of the
total number of steps taken each day.

ggplot(Total\_Steps, aes(x = steps)) + geom\_histogram(fill = ``blue'',
binwidth = 1000) + labs(title = ``Daily Steps'', x = ``Steps'', y =
``Frequency'')

\hypertarget{warning-removed-8-rows-containing-non-finite-values-stat_bin.}{%
\subsection{Warning: Removed 8 rows containing non-finite values
(stat\_bin).}\label{warning-removed-8-rows-containing-non-finite-values-stat_bin.}}

\#1. Calculate and report the mean and median of the total number of
steps taken per day Total\_Steps{[}, .(Mean\_Steps = mean(steps, na.rm =
TRUE), Median\_Steps = median(steps, na.rm = TRUE)){]}

\hypertarget{mean_steps-median_steps}{%
\subsection{Mean\_Steps Median\_Steps}\label{mean_steps-median_steps}}

\#\#1: 10766.19 10765

\#\#Make a time series plot (i.e.~𝚝𝚢𝚙𝚎 = ``𝚕'') of the 5-minute interval
(x-axis) and the average number of steps taken, averaged across all days
(y-axis) IntervalDT \textless- activityDT{[}, c(lapply(.SD, mean, na.rm
= TRUE)), .SDcols = c(``steps''), by = .(interval){]}

ggplot(IntervalDT, aes(x = interval , y = steps)) +
geom\_line(color=``blue'', size=1) + labs(title = ``Avg. Daily Steps'',
x = ``Interval'', y = ``Avg. Steps per day'')

\#\#Which 5-minute interval, on average across all the days in the
dataset, contains the maximum number of steps? IntervalDT{[}steps ==
max(steps), .(max\_interval = interval){]} \#\# max\_interval \#\# 1:
835

\#\#Calculate and report the total number of missing values in the
dataset (i.e.~the total number of rows with 𝙽𝙰s)
activityDT{[}is.na(steps), .N {]} \#\# {[}1{]} 2304 \# alternative
solution nrow(activityDT{[}is.na(steps),{]}) \#\# {[}1{]} 2304

\#\#Devise a strategy for filling in all of the missing values in the
dataset. The strategy does not need to be sophisticated. For example,
you could use the mean/median for that day, or the mean for that
5-minute interval, etc. \# Filling in missing values with median of
dataset. activityDT{[}is.na(steps), ``steps''{]} \textless-
activityDT{[}, c(lapply(.SD, median, na.rm = TRUE)), .SDcols =
c(``steps''){]} \#\#Create a new dataset that is equal to the original
dataset but with the missing data filled in. data.table::fwrite(x =
activityDT, file = ``data/tidyData.csv'', quote = FALSE) \#\#Make a
histogram of the total number of steps taken each day and calculate and
report the mean and median total number of steps taken per day. Do these
values differ from the estimates from the first part of the assignment?
What is the impact of imputing missing data on the estimates of the
total daily number of steps? \# total number of steps taken per day
Total\_Steps \textless- activityDT{[}, c(lapply(.SD, sum)), .SDcols =
c(``steps''), by = .(date){]}

\hypertarget{mean-and-median-total-number-of-steps-taken-per-day}{%
\section{mean and median total number of steps taken per
day}\label{mean-and-median-total-number-of-steps-taken-per-day}}

Total\_Steps{[}, .(Mean\_Steps = mean(steps), Median\_Steps =
median(steps)){]} \#\# Mean\_Steps Median\_Steps \#\# 1: 9354.23 10395

ggplot(Total\_Steps, aes(x = steps)) + geom\_histogram(fill = ``green'',
binwidth = 1000) + labs(title = ``Daily Steps'', x = ``Steps'', y =
``Frequency'')

\hypertarget{just-recreating-activitydt-from-scratch-then-making-the-new-factor-variable.-no-need-to-just-want-to-be-clear-on-what-the-entire-process-is.}{%
\section{Just recreating activityDT from scratch then making the new
factor variable. (No need to, just want to be clear on what the entire
process
is.)}\label{just-recreating-activitydt-from-scratch-then-making-the-new-factor-variable.-no-need-to-just-want-to-be-clear-on-what-the-entire-process-is.}}

activityDT \textless- data.table::fread(input = ``data/activity.csv'')
activityDT{[}, date := as.POSIXct(date, format = ``\%Y-\%m-\%d''){]}
activityDT{[}, \texttt{Day\ of\ Week}:= weekdays(x = date){]}
activityDT{[}grepl(pattern =
``Monday\textbar Tuesday\textbar Wednesday\textbar Thursday\textbar Friday'',
x = \texttt{Day\ of\ Week}), ``weekday or weekend''{]} \textless-
``weekday'' activityDT{[}grepl(pattern = ``Saturday\textbar Sunday'', x
= \texttt{Day\ of\ Week}), ``weekday or weekend''{]} \textless-
``weekend'' activityDT{[}, \texttt{weekday\ or\ weekend} :=
as.factor(\texttt{weekday\ or\ weekend}){]} head(activityDT, 10)

\hypertarget{steps-date-interval-day-of-week-weekday-or-weekend}{%
\subsection{steps date interval Day of Week weekday or
weekend}\label{steps-date-interval-day-of-week-weekday-or-weekend}}

\hypertarget{na-2012-10-01-0-monday-weekday}{%
\subsection{1: NA 2012-10-01 0 Monday
weekday}\label{na-2012-10-01-0-monday-weekday}}

\hypertarget{na-2012-10-01-5-monday-weekday}{%
\subsection{2: NA 2012-10-01 5 Monday
weekday}\label{na-2012-10-01-5-monday-weekday}}

\hypertarget{na-2012-10-01-10-monday-weekday}{%
\subsection{3: NA 2012-10-01 10 Monday
weekday}\label{na-2012-10-01-10-monday-weekday}}

\hypertarget{na-2012-10-01-15-monday-weekday}{%
\subsection{4: NA 2012-10-01 15 Monday
weekday}\label{na-2012-10-01-15-monday-weekday}}

\hypertarget{na-2012-10-01-20-monday-weekday}{%
\subsection{5: NA 2012-10-01 20 Monday
weekday}\label{na-2012-10-01-20-monday-weekday}}

\hypertarget{na-2012-10-01-25-monday-weekday}{%
\subsection{6: NA 2012-10-01 25 Monday
weekday}\label{na-2012-10-01-25-monday-weekday}}

\hypertarget{na-2012-10-01-30-monday-weekday}{%
\subsection{7: NA 2012-10-01 30 Monday
weekday}\label{na-2012-10-01-30-monday-weekday}}

\hypertarget{na-2012-10-01-35-monday-weekday}{%
\subsection{8: NA 2012-10-01 35 Monday
weekday}\label{na-2012-10-01-35-monday-weekday}}

\hypertarget{na-2012-10-01-40-monday-weekday}{%
\subsection{9: NA 2012-10-01 40 Monday
weekday}\label{na-2012-10-01-40-monday-weekday}}

\hypertarget{na-2012-10-01-45-monday-weekday}{%
\subsection{10: NA 2012-10-01 45 Monday
weekday}\label{na-2012-10-01-45-monday-weekday}}

\#\#Make a panel plot containing a time series plot (i.e.~𝚝𝚢𝚙𝚎 = ``𝚕'')
of the 5-minute interval (x-axis) and the average number of steps taken,
averaged across all weekday days or weekend days (y-axis). See the
README file in the GitHub repository to see an example of what this plot
should look like using simulated data. activityDT{[}is.na(steps),
``steps''{]} \textless- activityDT{[}, c(lapply(.SD, median, na.rm =
TRUE)), .SDcols = c(``steps''){]} IntervalDT \textless- activityDT{[},
c(lapply(.SD, mean, na.rm = TRUE)), .SDcols = c(``steps''), by =
.(interval, \texttt{weekday\ or\ weekend}){]}

ggplot(IntervalDT , aes(x = interval , y = steps,
color=\texttt{weekday\ or\ weekend})) + geom\_line() + labs(title =
``Avg. Daily Steps by Weektype'', x = ``Interval'', y = ``No.~of
Steps'') + facet\_wrap(\textasciitilde{}\texttt{weekday\ or\ weekend} ,
ncol = 1, nrow=2)

\end{document}
